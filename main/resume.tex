%============================================================================%
%
%	DOCUMENT DEFINITION
%
%============================================================================%

%we use article class because we want to fully customize the page and don't use a cv template
\documentclass[10pt,A4]{article}


%----------------------------------------------------------------------------------------
%	ENCODING
%----------------------------------------------------------------------------------------

% we use utf8 since we want to build from any machine
\usepackage[utf8]{inputenc}

%----------------------------------------------------------------------------------------
%	LOGIC
%----------------------------------------------------------------------------------------

% provides \isempty test
\usepackage{xstring, xifthen}

%----------------------------------------------------------------------------------------
%	FONT BASICS
%----------------------------------------------------------------------------------------

% some tex-live fonts - choose your own

%\usepackage[defaultsans]{droidsans}
%\usepackage[default]{comfortaa}
%\usepackage{cmbright}
\usepackage[default]{raleway}
%\usepackage{fetamont}
%\usepackage[default]{gillius}
%\usepackage[light,math]{iwona}
%\usepackage[thin]{roboto}

% set font default
\renewcommand*\familydefault{\sfdefault}
\usepackage[T1]{fontenc}

% more font size definitions
\usepackage{moresize}

%----------------------------------------------------------------------------------------
%	FONT AWESOME ICONS
%----------------------------------------------------------------------------------------

% include the fontawesome icon set
\usepackage{fontawesome}

% use to vertically center content
% credits to: http://tex.stackexchange.com/questions/7219/how-to-vertically-center-two-images-next-to-each-other
\newcommand{\vcenteredinclude}[1]{\begingroup
\setbox0=\hbox{\includegraphics{#1}}%
\parbox{\wd0}{\box0}\endgroup}

% use to vertically center content
% credits to: http://tex.stackexchange.com/questions/7219/how-to-vertically-center-two-images-next-to-each-other
\newcommand*{\vcenteredhbox}[1]{\begingroup
\setbox0=\hbox{#1}\parbox{\wd0}{\box0}\endgroup}

% icon shortcut
\newcommand{\icon}[3] {
	\makebox(#2, #2){\textcolor{maincol}{\csname fa#1\endcsname}}
}

% icon with text shortcut
\newcommand{\icontext}[4]{
	\vcenteredhbox{\icon{#1}{#2}{#3}}  \hspace{2pt}  \parbox{0.9\mpwidth}{\textcolor{#4}{#3}}
}

% icon with website url
\newcommand{\iconhref}[5]{
    \vcenteredhbox{\icon{#1}{#2}{#5}}  \hspace{2pt} \href{#4}{\textcolor{#5}{#3}}
}

% icon with email link
\newcommand{\iconemail}[5]{
    \vcenteredhbox{\icon{#1}{#2}{#5}}  \hspace{2pt} \href{mailto:#4}{\textcolor{#5}{#3}}
}

%----------------------------------------------------------------------------------------
%	PAGE LAYOUT  DEFINITIONS
%----------------------------------------------------------------------------------------

% page outer frames (debug-only)
% \usepackage{showframe}

% we use paracol to display breakable two columns
\usepackage{paracol}

% define page styles using geometry
\usepackage[a4paper]{geometry}

% remove all possible margins
\geometry{top=1cm, bottom=0cm, left=1cm, right=1cm}

\usepackage{fancyhdr}
\pagestyle{empty}

% space between header and content
\setlength{\headheight}{0pt}

% indentation is zero
\setlength{\parindent}{0mm}

%----------------------------------------------------------------------------------------
%	TABLE /ARRAY DEFINITIONS
%----------------------------------------------------------------------------------------

% extended aligning of tabular cells
\usepackage{array}

% custom column right-align with fixed width
% use like p{size} but via x{size}
\newcolumntype{x}[1]{%
>{\raggedleft\hspace{0pt}}p{#1}}%

%----------------------------------------------------------------------------------------
%	GRAPHICS DEFINITIONS
%----------------------------------------------------------------------------------------

%for header image
\usepackage{graphicx}

% use this for floating figures
% \usepackage{wrapfig}
% \usepackage{float}
% \floatstyle{boxed}
% \restylefloat{figure}

%for drawing graphics
\usepackage{tikz}
\usetikzlibrary{shapes, backgrounds, mindmap, trees}

%----------------------------------------------------------------------------------------
%	Color DEFINITIONS
%----------------------------------------------------------------------------------------
\usepackage{transparent}
\usepackage{color}

% primary color
\definecolor{maincol}{RGB}{ 225, 0, 0 }

% accent color, secondary
\definecolor{accentcol}{RGB}{ 250, 150, 10 }

% dark color
\definecolor{darkcol}{RGB}{ 70, 70, 70 }

% light color
\definecolor{lightcol}{RGB}{ 245, 245, 245 }

% Package for links, must be the last package used
\usepackage[hidelinks]{hyperref}

% returns minipage width minus two times \fboxsep
% to keep padding included in width calculations
% can also be used for other boxes / environments
\newcommand{\mpwidth}{\linewidth-\fboxsep-\fboxsep}

%============================================================================%
%
%	CV COMMANDS
%
%============================================================================%

%----------------------------------------------------------------------------------------
%	 CV LIST
%----------------------------------------------------------------------------------------

% renders a standard latex list but abstracts away the environment definition (begin/end)
\newcommand{\cvlist}[1] {
	\begin{itemize}{#1}\end{itemize}
}

%----------------------------------------------------------------------------------------
%	 CV TEXT
%----------------------------------------------------------------------------------------

% base class to wrap any text based stuff here. Renders like a paragraph.
% Allows complex commands to be passed, too.
% param 1: *any
\newcommand{\cvtext}[1] {
	\begin{tabular*}{1\mpwidth}{p{0.98\mpwidth}}
		\parbox{1\mpwidth}{#1}
	\end{tabular*}
}

%----------------------------------------------------------------------------------------
%	CV SECTION
%----------------------------------------------------------------------------------------

% Renders a a CV section headline with a nice underline in main color.
% param 1: section title
\newcommand{\cvsection}[1] {
	\vspace{8pt}
	\cvtext{
		\textbf{\LARGE{\textcolor{darkcol}{\uppercase{#1}}}}\\[-4pt]
		\textcolor{maincol}{ \rule{0.1\textwidth}{2pt} } \\
	}
}

%----------------------------------------------------------------------------------------
%	META SKILL
%----------------------------------------------------------------------------------------

% Renders a progress-bar to indicate a certain skill in percent.
% param 1: name of the skill / tech / etc.
% param 2: level (for example in years)
% param 3: percent, values range from 0 to 1
\newcommand{\cvskill}[3] {
	\begin{tabular*}{1\mpwidth}{p{0.72\mpwidth}  r}
 		\textcolor{black}{\textbf{#1}} & \textcolor{maincol}{#2}\\
	\end{tabular*}%

	\hspace{4pt}
	\begin{tikzpicture}[scale=1, rounded corners=2pt, very thin]
		\fill [lightcol] (0,0) rectangle (1\mpwidth, 0.15);
		\fill [maincol] (0,0) rectangle (#3\mpwidth, 0.15);
  	\end{tikzpicture}%
}

%----------------------------------------------------------------------------------------
%	 CV EVENT
%----------------------------------------------------------------------------------------

% Renders a table and a paragraph (cvtext) wrapped in a parbox (to ensure minimum content
% is glued together when a pagebreak appears).
% Additional information can be passed in text or list form (or other environments).
% the work you did
% param 1: time-frame i.e. Sep 14 - Jan 15 etc.
% param 2:	 event name (job position etc.)
% param 3: Customer, Employer, Industry
% param 4: Short description
% param 5: work done (optional)
% param 6: technologies include (optional)
% param 7: achievements (optional)
\newcommand{\cvevent}[7] {

	% we wrap this part in a parbox, so title and description are not separated on a pagebreak
	% if you need more control on page breaks, remove the parbox
	\parbox{\mpwidth}{
		\vspace{8pt}
		\begin{tabular*}{1\mpwidth}{p{0.72\mpwidth}  r}
	 		\textcolor{black}{\textbf{#2}} & \colorbox{maincol}{\makebox[0.25\mpwidth]{\textcolor{white}{#1}}} \\
			\textcolor{maincol}{\textbf{#3}} & \\
		\end{tabular*}\\[20pt]

		\ifthenelse{\isempty{#4}}{}{
			\cvtext{#4}\\
		}
	}

	\ifthenelse{\isempty{#5}}{}{
		\vspace{1pt}
		{#5}
	}

	\ifthenelse{\isempty{#6}}{}{
		\vspace{1pt}
		\cvtext{\textbf{Technologies include:}}\\
		{#6}
	}

	\ifthenelse{\isempty{#7}}{}{
		\vspace{-10pt}
		\cvtext{\textbf{Achievements include:}}\\
		{#7}
	}
	\vspace{10pt}
}

%----------------------------------------------------------------------------------------
%	 CV META EVENT
%----------------------------------------------------------------------------------------

% Renders a CV event on the sidebar
% param 1: title
% param 2: subtitle (optional)
% param 3: customer, employer, etc,. (optional)
% param 4: info text (optional)
\newcommand{\cvmetaevent}[4] {
	\textcolor{maincol} {\cvtext{\textbf{\begin{flushleft}#1\end{flushleft}}}}

	\ifthenelse{\isempty{#2}}{}{
	\textcolor{darkcol} {\cvtext{\textbf{#2}} }
	}

	\ifthenelse{\isempty{#3}}{}{
		\cvtext{{ \textcolor{darkcol} {#3} }}\\
	}

	\cvtext{#4}\\[14pt]
}

%---------------------------------------------------------------------------------------
%	QR CODE
%----------------------------------------------------------------------------------------

% Renders a qrcode image (centered, relative to the parentwidth)
% param 1: percent width, from 0 to 1
\newcommand{\cvqrcode}[1] {
	\begin{center}
		% \includegraphics[width={#1}\mpwidth]{qrcode}
	\end{center}
}

%============================================================================%
%
%
%
%	DOCUMENT CONTENT
%
%
%
%============================================================================%
\begin{document}
\columnratio{0.31}
\setlength{\columnsep}{2.2em}
\setlength{\columnseprule}{4pt}
\colseprulecolor{lightcol}
\begin{paracol}{2}
\begin{leftcolumn}
%---------------------------------------------------------------------------------------
%	IMAGE
%----------------------------------------------------------------------------------------
% \includegraphics[width=\linewidth]{profile-small.jpeg}	%trimming relative to image size

%---------------------------------------------------------------------------------------
%	SKILLS
%----------------------------------------------------------------------------------------
\cvsection{\faConnectdevelop \hspace{0.1cm} SKILLS}

\cvskill{Python} {5+ yrs} {1} \\[5pt]
\cvskill{Applied ML / CV} {5+ yrs} {1} \\[5pt]
\cvskill{Cloud Engineering} {5+ yrs} {1} \\[5pt]
\cvskill{MLOps} {5+ yrs} {1} \\[5pt]
\cvskill{Docker / Kubernetes} {3+ yrs} {0.8} \\[5pt]
\cvskill{Robotics} {3+ yrs} {0.72} \\[5pt]


%---------------------------------------------------------------------------------------
%	ACCOLADES
%----------------------------------------------------------------------------------------
\cvsection{\faBookmarkO \hspace{0.1cm} ACCOLADES}
\vspace{-30pt}
\cvmetaevent{}{US patent holder}{Invented a novel method to measure the internal temperature of cooked meat products.}{}\\[-52pt]
\cvmetaevent{---\vspace{-4pt}}{Member of Interpol's \\CACDevOps Group}{Invite-only annual meetup of engineers and international law enforcement agents to build tools to help solve human trafficking and child sex crimes.}{}\\[-52pt]
\cvmetaevent{---\vspace{-4pt}}{Featured guest on AWS \\Innovation Ambassadors \\podcast}{Interviewed by Sara Armstrong about building an enterprise-scale computer vision-based inventory tracking system.}{}\\[-50pt]
\cvmetaevent{---\vspace{-4pt}}{Honored Guest Speaker.}{}{}\\[-65pt]
\cvmetaevent{}{}{• AWS re:Invent 2021}{}\\[-73pt]
\cvmetaevent{}{}{• Sensors \& IIoT: Manufacturing, Automation \& Robotics 2021}{}

%---------------------------------------------------------------------------------------
%	EDUCATION
%----------------------------------------------------------------------------------------
% \newpage
\vspace{-30pt}
\cvsection{\faGraduationCap\hspace{0.1cm} EDUCATION}
\cvmetaevent
{\vspace{-15pt}2008 - 2011}
{Bachelor's Degree in \\Neuroscience \& Mathematics\vspace{0.2cm}}
{Temple University}
% \vfill\null

\vspace{-40pt}

%---------------------------------------------------------------------------------------
%	CONTACT
%----------------------------------------------------------------------------------------
\cvsection{\faUser\hspace{0.1cm}  CONTACT}
\vspace{-6pt}

\icontext{MapMarker}{12}{\hspace{-5pt}Austin, TX}{black}\\[4pt]
\icontext{MobilePhone}{12}{\hspace{-5pt}(908) 370-7440}{black}\\[4pt]
\iconemail{Envelope}{12}{\hspace{-5pt}jeremyegerard@gmail.com}{}{black}\\[4pt]
\icontext{Linkedin}{12}{\hspace{-5pt}linkedin.com/in/jeremygerard}{black}\\[4pt]
\icontext{Github}{12}{\hspace{-5pt}github.com/jeremy-gerard}{black}\\[4pt]

%----------------------------------------------------------------------------------------
% \vfill\null
% \cvqrcode{0.7}
% \vfill
% \cvqrcode{0.7}

% \newpage
% \mbox{} % hotfix to place qrcode on the bottom when there are not other elements
% \vfill
% \cvqrcode{0.7}

\end{leftcolumn}
\begin{rightcolumn}

%---------------------------------------------------------------------------------------
%	TITLE  HEADER
%----------------------------------------------------------------------------------------
\fcolorbox{white}{darkcol}{\begin{minipage}[c][1.75cm][c]{1\mpwidth}
	\begin {center}
		\HUGE{ \textbf{ \textcolor{white}{ \uppercase{ JEREMY GERARD } } } } \\[-24pt]
		% \textcolor{white}{ \rule{0.1\textwidth}{1.25pt} } \\[4pt]
		% \large{ \textcolor{white} {Machine Learning Engineer} }
	\end {center}
\end{minipage}} \\[14pt]
\vspace{-12pt}

%---------------------------------------------------------------------------------------
%	PROFILE
%----------------------------------------------------------------------------------------
% \vfill\null
% \cvsection{\faSquareO \hspace{0.1cm} PROFILE}
% \vspace{-10pt}
% \cvsection{\hspace{0.1cm}}
\vspace{8pt}
% \cvsection{PROFILE}
% indent
\hspace{4pt}
\cvtext{Cross-disciplined technologist and machine learning engineer with a \\neuroscience background seeking new opportunities in applied \\technology that demand a diverse range of knowledge and technical \\aptitude and adaptive critical thinking. I'm looking for a role where \\I can use my experience to make a meaningful contribution while also \\continuing to grow and learn at pace with the latest technologies.}


%---------------------------------------------------------------------------------------
%	WORK EXPERIENCE
%----------------------------------------------------------------------------------------
% \vfill\null
\vspace{20pt}
\cvsection{\faBriefcase \hspace{1pt} EXPERIENCE}
\vspace{-10pt}
\cvevent
	{2022 - 2023}
	{Clearview AI}
	{Machine Learning Engineer}
	{}
	{\vspace{-20pt}
		\cvlist{
			\item Devised and executed an enterprise-wide MLOps strategy, including\\standardization of production ecosystems, unit testing, and model \\optimization for CPU and GPU. \vspace{-4pt}
			\item Developed cutting-edge computer vision models for presentation attack detection and identity verification, leveraging ChatGPT for entity extraction from metadata. \vspace{-4pt}
			\item Delivered image enhancement and exposure correction algorithms to \\support investigators in facial recognition from poor-quality photos. \vspace{-4pt}
		}
	}
	{}
	{}

% \vfill\null
\cvevent
	{2018 - 2022}
	{Tyson Foods}
	{Lead Developer, Emerging Technology}
	% {Successfully architected and led the successful development and deployment of a diverse range of projects and soultions around ML, CV, hyperspectral imaging, IoT, robotics - most of which were recognized company-wide for their success in integrating new technologies ito the business for the first time, and enabling unprecedented improvements in business operations through data and automation}
	{}
	{\vspace{-20pt}
		\cvlist{
			\item Designed, developed, and implemented two end-to-end MLOps pipelines in AWS that autonomously handle image collection, annotation, curation, model training, evaluation, and deployment back to the edge. \vspace{-4pt}
			\item Contributed to a pending patent for a novel solution that measures the internal temperature of cooked food products using a robotic arm, a 3D linescan camera, and an infrared camera. \vspace{-4pt}
			\item Led the data science effort for the company-wide COVID-19 response, devising forecasting, risk evaluation, and transmission rate factor models, and developing a real-time interactive dashboard used by Tyson's ELT.
		}}
	{}
	{}
	% {\cvlist{
	% 	\item Spearheaded the architecture, development, and implementation of two separate end-to-end MLOps pipelines in AWS to autonomously handle image collection from edge to cloud, image annotation and curation, model training and evaluation, and model deployment back to the edge
	% 	\item Listed as a principal inventor on a pending patent for the design and architecture of a novel solution which autonomously measures internal temperature of cooked food products using a robotic arm, a 3D linescan camera, and an infrared camera
	% 	\item Acted as data science lead for company-wide COVID-19 response effort. Devised forecasting, risk evaluation, and transmission rate factro models and personally develop4d a NRT-interactive dashboard and front-end used by Tyson's ELT
	% }}
	% {}
	% {}

\cvevent
	{2016 - 2018}
	{Hewlett Packard Enterprise}
	{IT Systems Analyst, Marketing IT}
	{}
	{\vspace{-20pt}
		\cvlist{
			\item Served as the technical lead for front-end optimization of hpe.com, \\including DNS and traffic management support team. \vspace{-4pt}
			\item Managed a domain portfolio for enterprise-wide customer-facing webspace, and led the implementation of a CDN strategy that included FEO, global content delivery, caching, and security. \vspace{-4pt}
		}
	}
	{}
	{}

\cvevent
	{2009 - 2012}
	{Neural Instrumentation Lab, Temple University}
	{Research Associate, Project Manager}
	{}
	{\vspace{-20pt}
		\cvlist{
			\item Managed a project that developed an adaptive motor control system for a pneumatic robotic arm based on cerebellar neuronal microcircuitry, \\incorporating real-time motion-tracking using original machine learning and computer vision software in C++. \vspace{-4pt}
			\item Presented research at 33rd Annual International Conference of the IEEE Engineering in Medicine and Biology Society
		}}
	{}
	{}

	\vfill\null

\end{rightcolumn}
\end{paracol}
\end{document}

